\section{Project Summary and Team Evaluation}
\label{sec:team_evaluation}

This final section provides a comprehensive overview of the project's management, the team's collaborative process, and an evaluation of the overall workflow from inception to completion.

\subsection{Member Contributions and Task Allocation}
The project's success was built on a clear allocation of tasks and the dedicated contribution of each team member. The work was divided based on individual strengths and interests to maximize efficiency and quality.

\begin{description}
	\item[Member 1 - [Tên vai trò, ví dụ: Team Lead / Core Architect]]
	      [Mô tả chi tiết đóng góp, ví dụ: Responsible for the overall system architecture, designing the core domain models (Book, User, Loan), and implementing the Singleton and Strategy patterns. Led team meetings and managed the Git repository.]

	\item[Member 2 - [Tên vai trò, ví dụ: Backend Developer / Data Layer]]
	      [Mô tả chi tiết đóng góp, ví dụ: Developed the entire data persistence layer, including the CSVHandler for file I/O and the data integrity module using file hashing. Implemented the book and member management logic.]

	\item[Member 3 - [Tên vai trò, ví dụ: Backend Developer / Business Logic]]
	      [Mô tả chi tiết đóng góp, ví dụ: Focused on the business logic for loan management, including borrowing, returning, and overdue calculations. Implemented the Observer and Decorator patterns for the notification and book description features.]

	\item[Member 4 - [Tên vai trò, ví dụ: UI \& Documentation]]
	      [Mô tả chi tiết đóng góp, ví dụ: Developed the entire console-based user interface, ensuring a user-friendly and interactive experience. Was also responsible for writing and formatting the final technical report and creating the UML diagrams.]

\end{description}

\subsection{Development Workflow and Collaboration}
To ensure a smooth and organized development process, our team adopted a structured workflow utilizing industry-standard tools and practices.

\begin{itemize}
	\item \textbf{Version Control:} All source code was managed using \textbf{Git}, a distributed version control system that is the de facto standard in modern software development \cite{Chacon2014}. A central repository was hosted on \textbf{GitHub}, and the team followed a feature-branch workflow to facilitate parallel development and code review before integration.

	\item \textbf{Communication:} Primary communication was conducted through a dedicated \textbf{Discord} channel for daily updates and quick questions. We also held weekly online meetings to discuss progress, resolve blocking issues, and plan the next steps.

	\item \textbf{Task Management:} We utilized a simple Kanban-style board on \textbf{Trello} to track the status of all tasks (To Do, In Progress, Done). This provided clear visibility into the project's overall progress and helped identify bottlenecks early.
\end{itemize}

\subsection{Key Challenges and Resolutions}
Throughout the project, we encountered several challenges that tested our problem-solving and teamwork skills:

\begin{description}
	\item[Technical Design Decisions]
	      \textbf{Challenge:} Choosing the most suitable design patterns from the catalog presented by Gamma et al. \cite{GoF1994} was a point of extensive debate.
	      \textbf{Resolution:} The team held dedicated design sessions where we would whiteboard different approaches and discuss the pros and cons of each. The final decision was always made collectively to ensure everyone understood the chosen architecture.

	\item[Code Integration]
	      \textbf{Challenge:} As members completed their features, merging different Git branches occasionally resulted in code conflicts.
	      \textbf{Resolution:} We resolved this by adopting a policy of frequent communication. Before starting a major change, members would announce their plans to the team. Before merging, the developer would pull the latest changes from the main branch to resolve conflicts locally first.

	\item[Data Integrity Verification]
	      \textbf{Challenge:} Ensuring the integrity of data stored in plain text CSV files was a significant security concern.
	      \textbf{Resolution:} We implemented a mechanism to compute and store a cryptographic hash (using SHA-256) for each data file. Before reading any file, the system re-computes its hash and compares it to the stored value to detect any tampering, a fundamental technique in data security \cite{Stallings2017}.
\end{description}