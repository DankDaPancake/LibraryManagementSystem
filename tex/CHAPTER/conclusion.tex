\section{Conclusion}
\label{sec:conclusion}

This project has successfully culminated in the development of a functional and robust Electronic Library Management System. By adhering to the principles of Object-Oriented Programming and leveraging established software design patterns, the project has achieved its primary goal of creating a maintainable and extensible software solution for a real-world problem. The final application effectively demonstrates the practical application of key academic concepts in a complete software development lifecycle.

\subsection{Challenges Faced During Development}
The development process, while successful, presented several challenges that provided valuable learning experiences:

\begin{itemize}
    \item \textbf{Design Pattern Selection:} One of the initial challenges was selecting the most appropriate design patterns. It required careful analysis to determine which pattern best suited each specific problem, such as choosing the Strategy pattern for search functionality versus a simpler conditional approach.

    \item \textbf{Data Persistence Logic:} Implementing a reliable data handler for CSV files was more complex than anticipated. Ensuring data integrity through file hashing, handling potential file I/O exceptions, and managing data consistency across multiple files required careful planning and implementation.

    \item \textbf{Collaborative Workflow:} As a team project, managing the source code via version control (Git) occasionally led to merge conflicts. Establishing and adhering to a consistent coding style across all modules also required continuous communication and discipline.
\end{itemize}

\subsection{Future Enhancements}
The current system serves as a solid foundation that can be extended with numerous features to increase its value and utility. Potential directions for future work include:

\begin{itemize}
    \item \textbf{Graphical User Interface (GUI):} The most significant improvement would be to replace the current console-based interface with a user-friendly GUI. This could be developed using a framework like Qt for C++ or JavaFX/Swing for Java, which would dramatically improve the user experience.

    \item \textbf{Database Integration:} To enhance performance, scalability, and data integrity, migrating from CSV files to a relational database system is a logical next step. Using a lightweight database like SQLite or a more powerful one like MySQL would be a major architectural improvement.

    \item \textbf{Book Reservation System:} A valuable new feature would be to allow members to reserve a book that is currently on loan. The system could automatically notify the member once the book becomes available.

    \item \textbf{Advanced Reporting and Analytics:} The reporting module could be expanded to generate more detailed analytics, such as identifying the most popular books, tracking peak borrowing times, or creating activity reports for members.

    \item \textbf{REST API for Remote Access:} Exposing the system's core functionalities through a RESTful API would enable the development of web-based or mobile clients, allowing users to interact with the library from anywhere.
\end{itemize}

In summary, this project has not only met its specified requirements but has also provided a rich learning experience in software design and object-oriented methodologies, paving the way for future enhancements and development.